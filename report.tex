\documentclass[12pt,a4paper]{article}
\usepackage[UTF8]{ctex}
\usepackage{amsmath}
\usepackage{graphicx}
\usepackage{hyperref}
\usepackage{geometry}
\usepackage{float}
\usepackage{subfigure}
\usepackage{listings}
\usepackage{xcolor}
\usepackage{booktabs}

\geometry{left=2.5cm,right=2.5cm,top=2.5cm,bottom=2.5cm}

% 代码样式
\lstset{
    language=Python,
    basicstyle=\ttfamily\small,
    keywordstyle=\color{blue},
    commentstyle=\color{green!60!black},
    stringstyle=\color{red},
    numbers=left,
    numberstyle=\tiny\color{gray},
    frame=single,
    breaklines=true,
    captionpos=b
}

\title{Q-Learning算法在网格环境中的应用与实验分析}
\author{liucheng\\cliu425@connect.hkust-gz.edu.cn}
\date{\today}

\begin{document}

\maketitle

\section{引言}

\subsection{Q-Learning算法及其意义}

Q-Learning是一种无模型的强化学习算法,由Watkins在1989年提出\cite{watkins1989learning}。它是时序差分(Temporal Difference, TD)学习的一种,通过学习一个动作价值函数$Q(s,a)$来找到最优策略。

Q-Learning的核心更新公式为:
\begin{equation}
Q(s_t, a_t) \leftarrow Q(s_t, a_t) + \alpha \left[ r_{t+1} + \gamma \max_{a'} Q(s_{t+1}, a') - Q(s_t, a_t) \right]
\end{equation}

其中:
\begin{itemize}
    \item $\alpha$ 是学习率(learning rate),控制新信息对Q值的更新程度
    \item $\gamma$ 是折扣因子(discount factor),表示对未来奖励的重视程度
    \item $r_{t+1}$ 是执行动作后获得的即时奖励
    \item $\max_{a'} Q(s_{t+1}, a')$ 是下一状态的最大Q值
\end{itemize}

Q-Learning算法的重要意义在于:
\begin{enumerate}
    \item \textbf{无模型学习}:不需要环境的转移概率模型,可以直接从经验中学习
    \item \textbf{离线学习}:可以学习最优策略,而不需要遵循该策略(off-policy)
    \item \textbf{收敛性保证}:在满足一定条件下,Q-Learning可以保证收敛到最优Q函数
    \item \textbf{广泛应用}:适用于离散状态空间的决策问题,如路径规划、游戏AI等
\end{enumerate}

在本次实验中,我们使用Q-Learning算法解决一个9×7网格环境中的路径规划问题,agent需要从起点移动到终点,同时避开危险区域(Holes)。

\section{环境与状态描述}

\subsection{环境设置}

本实验使用的环境是基于Gymnasium的FrozenLake环境构建的自定义网格环境,具体参数如下:

\begin{itemize}
    \item \textbf{网格大小}:9列 × 7行(width = 9, height = 7)
    \item \textbf{坐标系}:使用用户友好的1-based坐标系统,左下角为原点(1,1),右上角为(9,7)
    \item \textbf{状态空间}:共有63个状态(9 × 7),每个状态对应网格中的一个位置
    \item \textbf{动作空间}:4个离散动作
    \begin{itemize}
        \item 0 = LEFT(向左)
        \item 1 = DOWN(向下)
        \item 2 = RIGHT(向右)
        \item 3 = UP(向上)
    \end{itemize}
\end{itemize}

\subsection{环境特性}

\subsubsection{Frozen Cells(危险区域)}

环境中共有14个Holes(危险区域),坐标为:
\begin{align*}
\text{Holes} = \{ &(2,4), (3,2), (3,4), (4,4), (5,2), (5,4),\\
                  &(6,2), (6,4), (7,1), (7,2), (7,4), (7,6),\\
                  &(7,7), (8,6) \}
\end{align*}

当agent尝试进入Hole时,会被阻止移动,停留在当前位置,并获得小的负奖励(-0.01)以鼓励agent避免这些区域。

\subsubsection{奖励机制}

\begin{itemize}
    \item 到达目标状态(Goal):奖励 +1.0
    \item 进入Hole(被阻止):奖励 -0.01
    \item 其他移动:奖励 0.0
\end{itemize}

\subsubsection{动作执行}

环境采用确定性动作执行(is\_slippery=False),即执行的动作会产生预期的结果,不存在随机性。

\section{实验配置}

\subsection{初始配置(Configuration 1)}

\begin{itemize}
    \item \textbf{起点(Start)}:$(6, 1)$
    \item \textbf{终点(Goal)}:$(6, 7)$
    \item \textbf{任务}:从位置(6,1)找到到达位置(6,7)的最优路径
\end{itemize}

这是一个相对简单的路径规划任务,起点和终点位于同一列,agent需要向上移动,同时避开中间的Holes。

\subsection{改变配置(Configuration 2)}

\begin{itemize}
    \item \textbf{起点(Position A)}:随机生成的安全位置
    \item \textbf{终点(Position C)}:随机生成的安全位置(与A不同)
    \item \textbf{任务}:从随机生成的起点A找到到达随机生成的终点C的最优路径
\end{itemize}

这种配置用于测试Q-Learning算法在不同起点和终点组合下的泛化能力和收敛性能。

\subsection{Q-Learning超参数}

实验中使用以下超参数配置:

\begin{table}[H]
\centering
\caption{Q-Learning超参数设置}
\begin{tabular}{lc}
\toprule
参数 & 值 \\
\midrule
学习率(Learning Rate, $\alpha$) & 0.8 \\
折扣因子(Discount Factor, $\gamma$) & 0.95 \\
初始探索率(Initial $\epsilon$) & 0.1 \\
探索率衰减($\epsilon$ Decay) & 0.995 \\
最小探索率($\epsilon$ Min) & 0.01 \\
训练轮数(Episodes) & 5000 \\
\bottomrule
\end{tabular}
\end{table}

\subsubsection{探索策略}

采用$\epsilon$-greedy策略进行动作选择:
\begin{equation}
a_t = \begin{cases}
\text{random action} & \text{with probability } \epsilon \\
\arg\max_{a} Q(s_t, a) & \text{with probability } 1-\epsilon
\end{cases}
\end{equation}

探索率$\epsilon$在训练过程中从0.1逐渐衰减到0.01,平衡探索与利用。

\section{Q表分析}

\subsection{初始配置的Q表}

初始配置下,训练完成后Q表的值反映了从起点(6,1)到终点(6,7)的最优路径的价值。

Q表维度为$63 \times 4$(63个状态,每个状态4个动作)。训练完成后:
\begin{itemize}
    \item 非零Q值表示agent在训练过程中探索过的状态-动作对
    \item Q值的大小反映了采取该动作后,能够获得的长期累积奖励
    \item 每个状态的最优动作对应Q值最大的动作
\end{itemize}

\begin{figure}[H]
\centering
\includegraphics[width=0.9\textwidth]{result/2025-1103-0021/qtable_initial.png}
\caption{初始配置的Q表可视化(按动作分类)}
\label{fig:qtable_initial}
\end{figure}

图\ref{fig:qtable_initial}展示了初始配置下Q表的热力图,四个子图分别对应LEFT、DOWN、RIGHT、UP四个动作。颜色越深表示Q值越大,表示该动作在该状态下的价值越高。

\subsection{改变配置的Q表}

改变配置后,由于起点和终点位置发生变化,Q表的值也会相应调整,反映出新的最优路径策略。

\begin{figure}[H]
\centering
\includegraphics[width=0.9\textwidth]{result/2025-1103-0021/qtable_changed.png}
\caption{改变配置的Q表可视化(按动作分类)}
\label{fig:qtable_changed}
\end{figure}

\subsection{Q表收敛前后对比}

\subsubsection{收敛前}

训练初期(episode 0),Q表初始化为全零:
\begin{equation}
Q(s, a) = 0, \quad \forall s, a
\end{equation}

此时agent主要依赖随机探索($\epsilon = 0.1$),Q值开始逐渐更新。

\subsubsection{收敛后}

训练完成后(episode 5000),Q表已经收敛到接近最优值:
\begin{itemize}
    \item Q值分布反映了从任意状态到目标状态的最优路径价值
    \item 靠近终点的状态具有较高的Q值
    \item Holes周围的Q值较低,表明agent学会了避开这些区域
\end{itemize}

两种配置的Q表都成功收敛,能够指导agent找到从起点到终点的最优路径。

\section{收敛性分析}

\subsection{初始配置的收敛过程}

\begin{figure}[H]
\centering
\includegraphics[width=0.9\textwidth]{result/2025-1103-0021/convergence_initial.png}
\caption{初始配置的收敛过程}
\label{fig:convergence_initial}
\end{figure}

图\ref{fig:convergence_initial}展示了初始配置下的训练过程:
\begin{itemize}
    \item \textbf{左图}:Q值收敛曲线。随着训练进行,Q表中的最大Q值逐渐增加并趋于稳定,表明算法正在学习到有效的策略。
    \item \textbf{右图}:Episode奖励曲线。随着训练进行,agent获得正奖励(到达终点)的频率逐渐增加,最终达到100\%的成功率。
\end{itemize}

从收敛曲线可以看出:
\begin{enumerate}
    \item 训练初期(episodes 0-500):Q值快速上升,agent快速学习基本策略
    \item 训练中期(episodes 500-2000):Q值继续增长,但增长速度减缓,agent在探索和利用之间平衡
    \item 训练后期(episodes 2000-5000):Q值趋于稳定,表明算法已经收敛到接近最优的策略
\end{enumerate}

\subsection{改变配置的收敛过程}

\begin{figure}[H]
\centering
\includegraphics[width=0.9\textwidth]{result/2025-1103-0021/convergence_changed.png}
\caption{改变配置的收敛过程}
\label{fig:convergence_changed}
\end{figure}

图\ref{fig:convergence_changed}展示了改变配置下的收敛过程。由于起点和终点位置的变化,收敛过程可能会有所不同:
\begin{itemize}
    \item 如果新的起点和终点之间的距离或路径复杂度不同,收敛速度可能不同
    \item 但整体趋势与初始配置类似:Q值逐渐增长并趋于稳定,成功率逐渐提高
\end{itemize}

\subsection{收敛性对比}

两种配置的收敛过程都表现出以下特点:
\begin{enumerate}
    \item \textbf{快速学习阶段}:前500个episodes内,agent快速学习基本的路径规划策略
    \item \textbf{优化阶段}:500-2000个episodes内,agent优化路径,提高成功率
    \item \textbf{稳定收敛阶段}:2000个episodes后,策略趋于稳定,成功率达到100\%
\end{enumerate}

高学习率($\alpha = 0.8$)使得算法能够快速更新Q值,而高折扣因子($\gamma = 0.95$)确保agent重视长期奖励,从而找到最优路径。

\section{最优路径分析}

\subsection{初始配置的最优路径}

\begin{figure}[H]
\centering
\includegraphics[width=0.8\textwidth]{result/2025-1103-0021/path_initial.png}
\caption{初始配置的最优路径可视化}
\label{fig:path_initial}
\end{figure}

图\ref{fig:path_initial}展示了初始配置下学习到的最优路径。从起点(6,1)到终点(6,7):
\begin{itemize}
    \item 蓝色线条表示最优路径
    \item 绿色方块表示起点(6,1)
    \item 黄色星形表示终点(6,7)
    \item 红色区域表示Holes(危险区域)
\end{itemize}

agent学会了避开Holes,选择安全且最短的路径到达终点。

\subsection{改变配置的最优路径}

\begin{figure}[H]
\centering
\includegraphics[width=0.8\textwidth]{result/2025-1103-0021/path_changed.png}
\caption{改变配置的最优路径可视化}
\label{fig:path_changed}
\end{figure}

图\ref{fig:path_changed}展示了改变配置下的最优路径。由于起点和终点位置随机生成,路径会有所不同,但agent仍然能够学习到避开Holes的最优路径。

\section{结果分析与观察}

\subsection{实验结果总结}

\begin{enumerate}
    \item \textbf{算法有效性}:Q-Learning算法成功学习到了从起点到终点的最优路径,在两种配置下都达到了100\%的成功率。
    
    \item \textbf{收敛性能}:算法在约2000个episodes后基本收敛,Q值趋于稳定,表明算法具有良好的收敛性。
    
    \item \textbf{路径规划能力}:学习到的策略能够有效避开Holes,选择安全且相对较短的路径到达目标。
    
    \item \textbf{泛化能力}:当起点和终点位置改变时,算法能够重新学习并适应新的环境配置。
\end{enumerate}

\subsection{关键观察}

\subsubsection{1. 探索与利用的平衡}

$\epsilon$-greedy策略在训练过程中发挥了重要作用:
\begin{itemize}
    \item 训练初期($\epsilon = 0.1$):较高的探索率使得agent能够充分探索环境,发现多种可能的路径
    \item 训练后期($\epsilon \approx 0.01$):较低的探索率使得agent主要利用已学到的策略,提高性能稳定性
\end{itemize}

\subsubsection{2. Q值的分布特征}

从Q表可视化可以看出:
\begin{itemize}
    \item 靠近终点的状态具有较高的Q值
    \item Holes周围的Q值较低,表明agent学会了惩罚进入危险区域的行为
    \item Q值呈现梯度分布,从终点向外逐渐减小,符合强化学习的价值传播机制
\end{itemize}

\subsubsection{3. 超参数的影响}

\begin{itemize}
    \item \textbf{高学习率(0.8)}:使得算法能够快速学习,但也可能导致Q值波动较大
    \item \textbf{高折扣因子(0.95)}:重视长期奖励,有助于找到真正的最优策略
    \item \textbf{探索率衰减}:平衡了训练初期的探索和后期的利用
\end{itemize}

\subsubsection{4. 环境特性对学习的影响}

\begin{itemize}
    \item \textbf{确定性动作}:由于is\_slippery=False,动作执行是确定的,这简化了学习过程
    \item \textbf{Hole处理机制}:阻止进入Hole并给予负奖励,帮助agent快速学会避开危险区域
    \item \textbf{状态空间大小}:63个状态相对较小,Q表可以完全存储,不会出现状态空间爆炸问题
\end{itemize}

\subsection{实验结论}

\begin{enumerate}
    \item Q-Learning算法在离散状态空间的路径规划问题中表现优异,能够有效学习最优策略。
    
    \item 超参数的选择对算法性能有重要影响。本实验中使用的高学习率和高折扣因子组合,适合当前的网格环境设置。
    
    \item 收敛速度取决于环境的复杂度和起点到终点的距离。简单的路径(如初始配置)收敛更快,复杂的路径可能需要更多episodes。
    
    \item 算法的鲁棒性良好。即使起点和终点位置改变,算法仍能在相同的训练轮数内收敛到最优策略。
    
    \item Q表作为策略的显式表示,具有很好的可解释性。通过可视化Q表,可以直观地理解agent的决策过程。
\end{enumerate}

\subsection{可能的改进方向}

\begin{itemize}
    \item \textbf{函数逼近}:对于更大的状态空间,可以考虑使用神经网络等函数逼近方法来替代Q表
    \item \textbf{经验回放}:引入经验回放机制可以提高样本利用效率
    \item \textbf{奖励塑形}:设计更细粒度的奖励函数,可以提供更多的学习信号
    \item \textbf{超参数优化}:通过网格搜索或贝叶斯优化等方法,找到更优的超参数组合
\end{itemize}

\section{结论}

本实验成功实现了Q-Learning算法在9×7网格环境中的应用。通过两个不同的配置(初始配置和改变配置),验证了算法的有效性和鲁棒性。实验结果表明:

\begin{itemize}
    \item Q-Learning能够有效学习最优路径规划策略
    \item 算法具有良好的收敛性能,在约2000个episodes后基本收敛
    \item 学习到的策略能够成功避开危险区域,找到安全且较短的路径
    \item 算法对不同的起点和终点配置具有良好的适应能力
\end{itemize}

这些结果为Q-Learning在实际路径规划问题中的应用提供了有价值的参考。

\begin{thebibliography}{9}
\bibitem{watkins1989learning}
Watkins, C. J. C. H., \& Dayan, P. (1992). Q-learning. \textit{Machine learning}, 8(3-4), 279-292.

\bibitem{sutton2018reinforcement}
Sutton, R. S., \& Barto, A. G. (2018). \textit{Reinforcement learning: An introduction}. MIT press.
\end{thebibliography}

\end{document}
